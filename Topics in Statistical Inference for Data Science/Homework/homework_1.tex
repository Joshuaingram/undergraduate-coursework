\PassOptionsToPackage{unicode=true}{hyperref} % options for packages loaded elsewhere
\PassOptionsToPackage{hyphens}{url}
%
\documentclass[
]{article}
\usepackage{lmodern}
\usepackage{amssymb,amsmath}
\usepackage{ifxetex,ifluatex}
\ifnum 0\ifxetex 1\fi\ifluatex 1\fi=0 % if pdftex
  \usepackage[T1]{fontenc}
  \usepackage[utf8]{inputenc}
  \usepackage{textcomp} % provides euro and other symbols
\else % if luatex or xelatex
  \usepackage{unicode-math}
  \defaultfontfeatures{Scale=MatchLowercase}
  \defaultfontfeatures[\rmfamily]{Ligatures=TeX,Scale=1}
\fi
% use upquote if available, for straight quotes in verbatim environments
\IfFileExists{upquote.sty}{\usepackage{upquote}}{}
\IfFileExists{microtype.sty}{% use microtype if available
  \usepackage[]{microtype}
  \UseMicrotypeSet[protrusion]{basicmath} % disable protrusion for tt fonts
}{}
\makeatletter
\@ifundefined{KOMAClassName}{% if non-KOMA class
  \IfFileExists{parskip.sty}{%
    \usepackage{parskip}
  }{% else
    \setlength{\parindent}{0pt}
    \setlength{\parskip}{6pt plus 2pt minus 1pt}}
}{% if KOMA class
  \KOMAoptions{parskip=half}}
\makeatother
\usepackage{xcolor}
\IfFileExists{xurl.sty}{\usepackage{xurl}}{} % add URL line breaks if available
\IfFileExists{bookmark.sty}{\usepackage{bookmark}}{\usepackage{hyperref}}
\hypersetup{
  pdftitle={Homework 1},
  pdfauthor={Joshua Ingram},
  pdfborder={0 0 0},
  breaklinks=true}
\urlstyle{same}  % don't use monospace font for urls
\usepackage[margin=1in]{geometry}
\usepackage{color}
\usepackage{fancyvrb}
\newcommand{\VerbBar}{|}
\newcommand{\VERB}{\Verb[commandchars=\\\{\}]}
\DefineVerbatimEnvironment{Highlighting}{Verbatim}{commandchars=\\\{\}}
% Add ',fontsize=\small' for more characters per line
\usepackage{framed}
\definecolor{shadecolor}{RGB}{248,248,248}
\newenvironment{Shaded}{\begin{snugshade}}{\end{snugshade}}
\newcommand{\AlertTok}[1]{\textcolor[rgb]{0.94,0.16,0.16}{#1}}
\newcommand{\AnnotationTok}[1]{\textcolor[rgb]{0.56,0.35,0.01}{\textbf{\textit{#1}}}}
\newcommand{\AttributeTok}[1]{\textcolor[rgb]{0.77,0.63,0.00}{#1}}
\newcommand{\BaseNTok}[1]{\textcolor[rgb]{0.00,0.00,0.81}{#1}}
\newcommand{\BuiltInTok}[1]{#1}
\newcommand{\CharTok}[1]{\textcolor[rgb]{0.31,0.60,0.02}{#1}}
\newcommand{\CommentTok}[1]{\textcolor[rgb]{0.56,0.35,0.01}{\textit{#1}}}
\newcommand{\CommentVarTok}[1]{\textcolor[rgb]{0.56,0.35,0.01}{\textbf{\textit{#1}}}}
\newcommand{\ConstantTok}[1]{\textcolor[rgb]{0.00,0.00,0.00}{#1}}
\newcommand{\ControlFlowTok}[1]{\textcolor[rgb]{0.13,0.29,0.53}{\textbf{#1}}}
\newcommand{\DataTypeTok}[1]{\textcolor[rgb]{0.13,0.29,0.53}{#1}}
\newcommand{\DecValTok}[1]{\textcolor[rgb]{0.00,0.00,0.81}{#1}}
\newcommand{\DocumentationTok}[1]{\textcolor[rgb]{0.56,0.35,0.01}{\textbf{\textit{#1}}}}
\newcommand{\ErrorTok}[1]{\textcolor[rgb]{0.64,0.00,0.00}{\textbf{#1}}}
\newcommand{\ExtensionTok}[1]{#1}
\newcommand{\FloatTok}[1]{\textcolor[rgb]{0.00,0.00,0.81}{#1}}
\newcommand{\FunctionTok}[1]{\textcolor[rgb]{0.00,0.00,0.00}{#1}}
\newcommand{\ImportTok}[1]{#1}
\newcommand{\InformationTok}[1]{\textcolor[rgb]{0.56,0.35,0.01}{\textbf{\textit{#1}}}}
\newcommand{\KeywordTok}[1]{\textcolor[rgb]{0.13,0.29,0.53}{\textbf{#1}}}
\newcommand{\NormalTok}[1]{#1}
\newcommand{\OperatorTok}[1]{\textcolor[rgb]{0.81,0.36,0.00}{\textbf{#1}}}
\newcommand{\OtherTok}[1]{\textcolor[rgb]{0.56,0.35,0.01}{#1}}
\newcommand{\PreprocessorTok}[1]{\textcolor[rgb]{0.56,0.35,0.01}{\textit{#1}}}
\newcommand{\RegionMarkerTok}[1]{#1}
\newcommand{\SpecialCharTok}[1]{\textcolor[rgb]{0.00,0.00,0.00}{#1}}
\newcommand{\SpecialStringTok}[1]{\textcolor[rgb]{0.31,0.60,0.02}{#1}}
\newcommand{\StringTok}[1]{\textcolor[rgb]{0.31,0.60,0.02}{#1}}
\newcommand{\VariableTok}[1]{\textcolor[rgb]{0.00,0.00,0.00}{#1}}
\newcommand{\VerbatimStringTok}[1]{\textcolor[rgb]{0.31,0.60,0.02}{#1}}
\newcommand{\WarningTok}[1]{\textcolor[rgb]{0.56,0.35,0.01}{\textbf{\textit{#1}}}}
\usepackage{graphicx,grffile}
\makeatletter
\def\maxwidth{\ifdim\Gin@nat@width>\linewidth\linewidth\else\Gin@nat@width\fi}
\def\maxheight{\ifdim\Gin@nat@height>\textheight\textheight\else\Gin@nat@height\fi}
\makeatother
% Scale images if necessary, so that they will not overflow the page
% margins by default, and it is still possible to overwrite the defaults
% using explicit options in \includegraphics[width, height, ...]{}
\setkeys{Gin}{width=\maxwidth,height=\maxheight,keepaspectratio}
\setlength{\emergencystretch}{3em}  % prevent overfull lines
\providecommand{\tightlist}{%
  \setlength{\itemsep}{0pt}\setlength{\parskip}{0pt}}
\setcounter{secnumdepth}{-2}
% Redefines (sub)paragraphs to behave more like sections
\ifx\paragraph\undefined\else
  \let\oldparagraph\paragraph
  \renewcommand{\paragraph}[1]{\oldparagraph{#1}\mbox{}}
\fi
\ifx\subparagraph\undefined\else
  \let\oldsubparagraph\subparagraph
  \renewcommand{\subparagraph}[1]{\oldsubparagraph{#1}\mbox{}}
\fi

% set default figure placement to htbp
\makeatletter
\def\fps@figure{htbp}
\makeatother


\title{Homework 1}
\author{Joshua Ingram}
\date{8/30/2020}

\begin{document}
\maketitle

\hypertarget{problem-1}{%
\section{Problem 1}\label{problem-1}}

\hypertarget{section}{%
\subsection{1.}\label{section}}

\begin{Shaded}
\begin{Highlighting}[]
\NormalTok{fit1_chile <-}\StringTok{ }\KeywordTok{multinom}\NormalTok{(vote }\OperatorTok{~}\StringTok{ }\NormalTok{age }\OperatorTok{+}\StringTok{ }\NormalTok{sex }\OperatorTok{+}\StringTok{ }\NormalTok{region }\OperatorTok{+}\StringTok{ }\NormalTok{statusquo, }\DataTypeTok{data =}\NormalTok{ chile)}
\end{Highlighting}
\end{Shaded}

\begin{verbatim}
## # weights:  36 (24 variable)
## initial  value 3490.689201 
## iter  10 value 2282.918477
## iter  20 value 2154.749143
## iter  30 value 2122.914761
## final  value 2122.850427 
## converged
\end{verbatim}

\begin{Shaded}
\begin{Highlighting}[]
\KeywordTok{summary}\NormalTok{(fit1_chile)}
\end{Highlighting}
\end{Shaded}

\begin{verbatim}
## Call:
## multinom(formula = vote ~ age + sex + region + statusquo, data = chile)
## 
## Coefficients:
##   (Intercept)         age       sexM   regionM     regionN    regionS
## N  -0.1295192 0.005397365  0.6993290 0.8411402 -0.30822679 0.33434661
## U   0.1589056 0.030373441 -0.2879216 1.3387552 -0.73715898 0.09402906
## Y  -0.3884494 0.025373911 -0.1039942 1.5077273  0.08062193 0.41435538
##     regionSA  statusquo
## N -0.0860732 -1.8230660
## U  0.0771308  0.3338119
## Y  0.2254778  1.8756710
## 
## Std. Errors:
##   (Intercept)         age      sexM   regionM   regionN   regionS  regionSA
## N   0.3063535 0.006410852 0.1735622 0.7920856 0.2866417 0.2554054 0.2268856
## U   0.2941819 0.006311186 0.1734192 0.7611136 0.2954569 0.2502714 0.2252501
## Y   0.3119872 0.006577450 0.1807100 0.7742880 0.2971859 0.2593320 0.2420183
##   statusquo
## N 0.1318171
## U 0.1069452
## Y 0.1207042
## 
## Residual Deviance: 4245.701 
## AIC: 4293.701
\end{verbatim}

``will abstain'' is the baseline category.

\hypertarget{section-1}{%
\subsection{2.}\label{section-1}}

\$\$ Y\_i\sim\emph{\{ind.\} Multinomial(p}\{i,1\}, p\_\{i,2\},
p\_\{i,3\}, p\_\{i,4\}) \textbackslash{}

\text{Where, 1 = "will vote no", 2 = "undecided", 3 = "will vote yes", 4 = "will abstain"}
\textbackslash{}

log(\frac{p_{i,j}}{p_{i,4}}) = \beta\emph{\{0,j\} + \beta}\{1,j\}age\_i
+ \beta\emph{\{2,j\}sex}\{i\} + \beta\emph{\{3,j\}region\_i +
\beta}\{4,j\}statusquo\_i, ; j = 1, 2, 3 \textbackslash{}

p\_\{i,4\} = 1 - \sum\^{}\{3\}\emph{\{j=1\} p}\{i,j\} \$\$

\hypertarget{section-2}{%
\subsection{3.}\label{section-2}}

We use the Likelihood Ratio Test (LRT)/Analysis of Deviance to test each
predictor ``as a whole.''

Example of hypothesis test with age (\(\beta_1\)):

\(H_0 \; : \; \beta_{1,1} = \beta_{1,2} = \beta_{1,3} =\beta_{1,4} = 0\)

\(H_A \; : \; \{\exists \; \beta_{1,j} \neq 0\}, \; j = 1, 2, 3, 4\)

\begin{Shaded}
\begin{Highlighting}[]
\KeywordTok{Anova}\NormalTok{(fit1_chile)}
\end{Highlighting}
\end{Shaded}

\begin{verbatim}
## Analysis of Deviance Table (Type II tests)
## 
## Response: vote
##           LR Chisq Df Pr(>Chisq)    
## age          41.55  3  4.998e-09 ***
## sex          59.22  3  8.627e-13 ***
## region       30.92 12   0.002028 ** 
## statusquo  1910.56  3  < 2.2e-16 ***
## ---
## Signif. codes:  0 '***' 0.001 '**' 0.01 '*' 0.05 '.' 0.1 ' ' 1
\end{verbatim}

\hypertarget{section-3}{%
\subsection{4.}\label{section-3}}

\hypertarget{a.}{%
\subsubsection{a.}\label{a.}}

\hypertarget{b.}{%
\subsubsection{b.}\label{b.}}

\hypertarget{c.}{%
\subsubsection{c.}\label{c.}}

\hypertarget{d.}{%
\subsubsection{d.}\label{d.}}

\hypertarget{section-4}{%
\subsection{5.}\label{section-4}}

\hypertarget{problem-2}{%
\section{Problem 2}\label{problem-2}}

\hypertarget{section-5}{%
\subsection{1.}\label{section-5}}

\hypertarget{section-6}{%
\subsection{2.}\label{section-6}}

\hypertarget{section-7}{%
\subsection{3.}\label{section-7}}

\hypertarget{section-8}{%
\subsection{4.}\label{section-8}}

\hypertarget{section-9}{%
\subsection{5.}\label{section-9}}

\hypertarget{problem-3}{%
\section{Problem 3}\label{problem-3}}

\hypertarget{section-10}{%
\subsection{1.}\label{section-10}}

\hypertarget{section-11}{%
\subsection{2.}\label{section-11}}

\hypertarget{section-12}{%
\subsection{3.}\label{section-12}}

\end{document}
